% This file was converted to LaTeX by Writer2LaTeX ver. 1.4
% see http://writer2latex.sourceforge.net for more info
\documentclass{article}
\usepackage{hyperref}
\makeatletter
\newcommand\arraybslash{\let\\\@arraycr}
\makeatother
\setlength\tabcolsep{1mm}
\renewcommand\arraystretch{1.3}
\title{BIO 144 - Phylogenetics}
\begin{document}
\maketitle
\date{}
\noindent Professor: Emily Jane McTavish\\
Office: SE1 284, Email: ejmctavish@ucmerced.edu\\
Course website: catcourses page\\

\noindent Teaching Assistant: Jasper Toscani Field\\
Email: jtoscanifield@ucmerced.edu\\

\noindent Lecture: Mondays 1:30-2:45 COB1 267\\
Wednesdays 1:30-2:45 COB1 267 (Lecture) or 1:30 -2:45 KL 202 (Lab)\\
Discussion: Thursdays 12:30 - 1:20 COB2 Bldg 267 (same room number, different building as Mondays)

\noindent Office Hours:\\
Dr. McTavish Wednesdays 3:30 - 4:30 SE1 282\\
TA, Jasper Toscani Field: Mondays 3:30 - 5:00 SE1 284





\section*{Introduction and Course Goals:}

This is a course on phylogenetics covering theory, methods, statistics and practice.
We will study the evolutionary models used to construct phylogenies (evolutionary trees), 
how these phylogenetic estimates may be used to understand evolutionary processes.
We will develop our computational skills to allow us analyze evolutionary data sets and perform phylogenetic analyses.
The course provides hands-on experience with several important phylogenetic software packages.
By the end of the course students should be able to understand much of the primary literature on modern phylogenetic methods and
know how to apply these methods.


\section*{Textbook}
None required.\\
Readings will be posted on catcourses. Some excerpts from \emph{Paul Lewis, Phylogenetics (Sinuaer) Forthcoming}
will be provided.

\section*{Learning outcomes}
By the end of this course you will:
\begin{itemize}
 \item Understand the evolutionary processes that generate phylogenetic trees
 \item Read and understand phylogentic trees
 \item Apply evolutionary theory to computational phylogenetic analyses
 \item Assess and discuss primary research papers with a phylogenetic focus
\end{itemize}

\section*{Assessment}
\begin{itemize}
 \item[25\%] Attendance and participation
 \item[15\%] Midterm (Wednesday, March 7, 2018, 1:30 pm)
 \item[30\%] Lab assignments and homeworks
 \item[20\%] Final exam (Saturday, May 05, 2018, 11:30 am)
\end{itemize}

\subsection*{Attendance and participation}
Come to class and discussion prepared, with assigned readings completed.
Participate in paper and research discussions, and ask questions when confused.
In class assignments will be included in participation score.
Attendance will be taken at lecture and lab. 
You may miss a total of 3 course meetings (lecture, lab or discussion) over the semester without penalization.

\subsection*{Lab assignments and Homeworks}
Labs and or homeworks will be assigned most weeks, and will be turned in via the catcourses class site. Late assignements will have 25\% deducted.
The homework assignments should be worked on your own.
This a 4 credit course which counts for the BIO EEB upper division laboratory requirement. 
You should expect to be spending on average 16 hours per week on this course, (12 for completing lab assignments and preparation in addition to the 4 contact hours per week)


\subsection*{Topics covered}
What are phylogenies and where to they come from?\\
How to read trees, and make evolutionary inferences from them\\
Classification and phylogeny\\
Models of evolution, expectations for morphological and molecular data\\
Alignment and tree inference from molecular data\\
Maximum likelihood and Bayesian statistics\\
Tree inference from morphological data\\
Phylogenetic dating\\
Biogeography\\
Comparative methods\\
Diversification and divergence\\
Microbial phylogenetics\\
Epidemiology and other applications\\


\section*{Course Policies}
\begin{itemize}
 \item[1.] Classroom interaction. Personal views and critical inquiry based on the material and topics at hand should be shared.
 Equally, I expect that the viewpoints of others will be respected. 
 Consider this course to be valuable practice to engage with your peers through professional communication and scholarly discourse. 
 \item[2.] Special accommodations. Students who need special accommodations are required to submit the form to me in person, preferably outside of class (e.g. office hours) within the first two weeks of the quarter. 
 If you will be requesting academic accommodations, you must first contact the Disability Services (http://disabilityservices.ucmerced.edu/) to make arrangements.
 If a personal situation arises that will result in an inability to complete an assignment on time, please contact us BEFORE the deadline and we will make reasonable accommodations.
\item[3.] Academic integrity. The University has established codes concerning proper academic conduct and the consequences resulting from improper behavior. 
Please be aware of these policies. The official UC Conduct Standards can be found at: http://studentlife.ucmerced.edu/content/uc-conduct-standards
\item[4.] Life as a UC-Merced Student. There are many campus services specifically suited to help you throughout your university career, please take advantage of your resources, 
including: University academic advising (http://advising.ucmerced.edu/), Health Services (http://health.ucmerced.edu/), and University Counseling and Psychological Services (http://counseling.ucmerced.edu/).
\end{itemize}


\end{document}
