\documentclass{beamer}
\useoutertheme{infolines}
\usepackage[utf8]{inputenc}
\usepackage{default}
%\usepackage{enumitem}
\usepackage{graphicx}
\usepackage[normalem]{ulem}
\newcommand\redout{\bgroup\markoverwith
{\textcolor{red}{\rule[.5ex]{2pt}{2pt}}}\ULon}
\usepackage{xcolor}
\usepackage{textpos}
\usepackage{tikz}   
\usepackage[round]{natbib}
\bibliographystyle{apalike}
\usepackage[T1]{fontenc}
\usepackage{pdfpages}
\usepackage{hyperref}


%\addtobeamertemplate{frametitle}{}{%
%\begin{textblock*}{100mm}(.8\textwidth,-1cm)
%\includegraphics[height=1cm,width=2cm]{opentreelogogrey}
%\end{textblock*}}

\useinnertheme{rectangles}
\usetheme{Copenhagen}
\setbeamertemplate{footline}[frame number]{}
\setbeamertemplate{footline}{}

\title[*]{Likelihood and models of evolution}
\author[*]{Emily Jane McTavish}
\institute[*]{
Life and Environmental Sciences\\
University of California, Merced\\
\texttt{ejmctavish@ucmerced.edu, twitter:snacktavish}\\
}
\date{}
\setbeamertemplate{itemize items}[triangle]

\begin{document}

\begin{frame}
\titlepage
(With thanks to Paul Lewis for slides) 
\end{frame}


\setbeamercolor{background canvas}{bg=}
\includepdf[pages=6]{/home/ejmctavish/Desktop/old_classes/QSB244/markphylo/2017/lec4-ML-CompatPars.pdf}

\begin{frame}
\textbf{Character conflict}
\begin{itemize}
 \item Two characters are compatible if they can both be mapped on the same
tree so that all of the character states displayed could be homologous.
 \item Incompatible characters are evidence of homoplasy in the data
 \item Homoplasy literally means the “same change” has occurred more than once
in the evolutionary history of the group.
\item The presence of homoplasy undermines analyses which rely on counting and minimziing changes.
\end{itemize}
\end{frame}



\begin{frame}
\textbf{How can we deal with character conflict?}
\begin{itemize}
 \item We need to apply an error model
 \item Likelihood provides a measure of surprise under different models
\end{itemize}
\end{frame}


\setbeamercolor{background canvas}{bg=}
\includepdf[pages=15-20]{/home/ejmctavish/Desktop/old_classes/QSB244/Lewis/Likelihood2017.pdf}

\setbeamercolor{background canvas}{bg=}
\includepdf[pages=22]{/home/ejmctavish/Desktop/old_classes/QSB244/Lewis/Likelihood2017.pdf}

\setbeamercolor{background canvas}{bg=}
\includepdf[pages=26-28]{/home/ejmctavish/Desktop/old_classes/QSB244/Lewis/Likelihood2017.pdf}


\begin{frame}
 \includegraphics[width=\textwidth]{/home/ejmctavish/Desktop/old_classes/QSB244/GradPhylo/docs/slides/ratetime}
\end{frame}

\setbeamercolor{background canvas}{bg=}
\includepdf[pages=31-32]{/home/ejmctavish/Desktop/old_classes/QSB244/Lewis/Likelihood2017.pdf}

\setbeamercolor{background canvas}{bg=}
\includepdf[pages=35]{/home/ejmctavish/Desktop/old_classes/QSB244/Lewis/Likelihood2017.pdf}


\setbeamercolor{background canvas}{bg=}
\includepdf[pages=7-8]{/home/ejmctavish/Desktop/old_classes/QSB244/markphylo/parsSummaryModels.pdf}


\setbeamercolor{background canvas}{bg=}
\includepdf[pages=36-38]{/home/ejmctavish/Desktop/old_classes/QSB244/Lewis/Likelihood2017.pdf}


\setbeamercolor{background canvas}{bg=}
\includepdf[pages=45]{/home/ejmctavish/Desktop/old_classes/QSB244/Lewis/Likelihood2017.pdf}

\setbeamercolor{background canvas}{bg=}
\includepdf[pages=49]{/home/ejmctavish/Desktop/old_classes/QSB244/Lewis/Likelihood2017.pdf}





\end{document}
