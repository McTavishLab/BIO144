\documentclass{beamer}
\useoutertheme{infolines}
\usepackage[utf8]{inputenc}
\usepackage{default}
%\usepackage{enumitem}
\usepackage{graphicx}
\usepackage[normalem]{ulem}
\newcommand\redout{\bgroup\markoverwith
{\textcolor{red}{\rule[.5ex]{2pt}{2pt}}}\ULon}
\usepackage{xcolor}
\usepackage{textpos}
\usepackage{tikz}   
\usepackage[round]{natbib}
\bibliographystyle{apalike}
\usepackage[T1]{fontenc}
\usepackage{pdfpages}
\usepackage{hyperref}

\graphicspath{{/home/ejmctavish/Desktop/Talks/}}



%\addtobeamertemplate{frametitle}{}{%
%\begin{textblock*}{100mm}(.8\textwidth,-1cm)
%\includegraphics[height=1cm,width=2cm]{opentreelogogrey}
%\end{textblock*}}

\useinnertheme{rectangles}
\usetheme{Copenhagen}
\setbeamertemplate{footline}[frame number]{}
\setbeamertemplate{footline}{}

\title[*]{Evolutionary background}
\author[*]{Emily Jane McTavish}
\institute[*]{
Life and Environmental Sciences\\
University of California, Merced\\
\texttt{ejmctavish@ucmerced.edu, twitter:snacktavish}\\
}
\date{}
\setbeamertemplate{itemize items}[triangle]

\begin{document}

\begin{frame}
\titlepage
\end{frame}



\begin{frame}
\begin{center}
 \Large{Phylogenies describe shared ancestry\\
and\\
inform our understanding of evolutionary processes}
\end{center}
\end{frame}


\begin{frame}
\begin{center}
\centerline{\includegraphics[width=\textwidth]{originofspeciescover}}
\end{center}
%TODO examples!
\end{frame}
%My interests are in understanding the tree of life

\begin{frame}
``. . . a naturalist, reflecting on the mutual affinities of organic beings, on their embryological relations, 
their geographical distribution, geological succession, and other such facts, might come to the conclusion that each species had not been independently created, but had descended . . . from other species.
Nevertheless, such a conclusion . . . would be unsatisfactory, until it could be shown HOW the innumerable species inhabiting this world have been modified . . .''\\
Charles Darwin, On the Origin of Species
\end{frame}


\begin{frame}
Darwin’s four postulates explain why/how evolution occurs:
\begin{itemize}
 \item Individuals within populations are variable
 \item Some of these variations are passed onto offspring
 \item Not all individuals produce the same number of offspring
 \item Individuals with certain heritable traits produce more offspring
\end{itemize}
\end{frame}

\begin{frame}
Alfred Russel Wallace was working at the same time, and studied geographic context of speciation
\end{frame}


\setbeamercolor{background canvas}{bg=}
\includepdf[pages=47-48]{/home/ejmctavish/Desktop/old_classes/bio141McTavish/slides/L2_OriginsEvolThought}



\begin{frame}
``the Ratio of Increase so high as to lead to a Struggle for Life, and as a consequence to Natural Selection, entailing Divergence of Character and the Extinction of less-improved forms. 
Thus, from the war of nature, from famine and death, the most exalted object which we are capable of conceiving, namely, the production of the higher animals, directly follows. 
There is grandeur in this view of life, with its several powers, having been originally breathed into a few forms or into one; 
 and that, whilst this planet has gone cycling on according to the fixed law of gravity, 
 from so simple a beginning endless forms most beautiful and most wonderful have been, and are being, evolved. ''
 Charles Darwin, On the Origin of Species
\end{frame}

\begin{frame}
\begin{center}
\centerline{\includegraphics[height=\textheight]{ithink}}
\end{center}
%TODO examples!
\end{frame}


\begin{frame}
 \Large{But what are species?}
\end{frame}

%-----comp meth examples------
\setbeamercolor{background canvas}{bg=}
\includepdf[pages=1-18]{/home/ejmctavish/Desktop/old_classes/bio141McTavish/slides/minispp}

\begin{frame}
\begin{center}
\centerline{\includegraphics[width=\textwidth]{dogdomestication}}
\end{center}
%TODO examples!
\end{frame}

%-----comp meth examples------
\setbeamercolor{background canvas}{bg=}
\includepdf[pages=26-28]{/home/ejmctavish/Desktop/old_classes/bio141McTavish/slides/L23_Ext&EvolTrends}

%-----comp meth examples------
\setbeamercolor{background canvas}{bg=}
\includepdf[pages=8]{/home/ejmctavish/Desktop/old_classes/bio141McTavish/slides/L23_Ext&EvolTrends}



\end{document}