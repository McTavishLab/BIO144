\documentclass{beamer}
\useoutertheme{infolines}
\usepackage[utf8]{inputenc}
\usepackage{default}
\usepackage{enumitem}
\usepackage{graphicx}
\graphicspath{{/home/ejmctavish/Desktop/Talks/}}

\usepackage[normalem]{ulem}
\newcommand\redout{\bgroup\markoverwith
{\textcolor{red}{\rule[.5ex]{2pt}{2pt}}}\ULon}
\usepackage{xcolor}
\usepackage{textpos}
\usepackage{tikz}   
\usepackage[round]{natbib}
\bibliographystyle{apalike}
\usepackage[T1]{fontenc}
\usepackage{booktabs}
\usepackage{pdfpages}


%\addtobeamertemplate{frametitle}{}{%
%\begin{textblock*}{100mm}(.8\textwidth,-1cm)
%\includegraphics[height=1cm,width=2cm]{opentreelogogrey}
%\end{textblock*}}

\usetheme{Copenhagen}
\setbeamertemplate{footline}[frame number]{}
\setbeamertemplate{footline}{}

\title[*]{Introduction to phylogenetics}
\author[*]{Emily Jane McTavish}

\begin{document}

\begin{frame}
\titlepage
\end{frame}


\begin{frame}
\begin{center}
\centerline{\includegraphics[scale=0.4]{tol_big}}
\end{center}
%TODO examples!
\tiny{Image Ethan Hein}
\end{frame}
%My interests are in understanding the tree of life


\begin{frame}
\begin{center}
\centerline{\includegraphics[scale=0.25]{tol_big_example}}
\end{center}
\tiny{Image Ethan Hein}
\end{frame}



\begin{frame}
\begin{center}
``Nothing in Biology Makes Sense Except in the Light of Evolution''\\
Theodosius Dobzhansky 
\end{center}
\end{frame}


\begin{frame}
\begin{center}
``Nothing in Biology Makes Sense Except in the \redout{Light of Evolution} \textcolor{red}{Context of Phylogenetics!}''\\ 
\end{center}
\end{frame}


\begin{frame}
\frametitle{Why use phylogenies?}
For an ecologist wants to know the relationships of the taxa in their study site

\centerline{\includegraphics[scale=0.3]{phylohomog}}

(example from the Cavender-Bares lab webpage)

\end{frame}

\begin{frame}
\frametitle{Why use phylogenies?}
To understand rates and types of evolutionary transitions

\centerline{\includegraphics[scale=1.75]{plantsex}}

%Tree structure is derived from taxonomy, where each tip represents all species in a single genus. Diploid chromosome number is indicated by the height of the innermost ring; all other rings indicate the presence or absence of the trait named at the base of the ring. 
%The ‘Other’ ring includes the states: apomictic, gynomonoecy, andromonoecy, polygamodioecy, and polygamomonoecy. 
%The sexual trait data displayed in the rings is based on 11,038 plant entries.
\tiny{The Tree of Sex Consortium,(2014) Scientific Data}
\end{frame}



\begin{frame}
Tracing epidemics
\centerline{\includegraphics[scale=0.32]{zikaa}}
\tiny{Faria et al., Science. 2016}
\end{frame}

\begin{frame}
\frametitle{Introduction}
Tracing epidemics, placing outbreaks in context
\centerline{\includegraphics[scale=0.4]{zikab}}
\tiny{Faria et al., Science. 2016}
\end{frame}


\begin{frame}
\center{The quantity of available sequence data for inferring evolutionary relationships is increasing rapidly}
\centerline{\includegraphics[scale=2]{seqfig}}
\tiny{http://genome.wellcome.ac.uk/}
\end{frame}


\begin{frame}
\begin{center}
Analysis of these data is computationally challenging\\
\bigskip
\Large{\emph{
``With the advent of modern molecular biology, the ability to collect biological sequence data has out-paced the ability to adequately analyze these data}''\\-- Jeff Thorne (Evolutionary biologist)\\}
\end{center}
\pause
\bigskip
Thorne et al., Journal of Molecular Evolution. \textbf{1991}
\end{frame}

\begin{frame}
Dramatic increase DNA sequence data generation has increased computational challenges\\
\centerline{\includegraphics[scale=0.4]{seqfig_arrow}}
\tiny{http://genome.wellcome.ac.uk/}
\end{frame}

\begin{frame}
\begin{center}
Inappropriate statistical models lead to incorrect conclusions\\
\end{center}
\end{frame}

\setbeamercolor{background canvas}{bg=}
\includepdf[pages=20]{WoodsHoleMole2017.pdf}


\begin{frame}
\textbf{How can we take advantage of evolutionary information in large data sets,}\\
in a way that is:
\begin{itemize}
 \item[-] biologically informed
 \item[-] statistically rigorous
 \item[-] computationally efficient?
\end{itemize}
\end{frame}


\begin{frame}
Goals for this course:
\begin{itemize}
 \item Understand the theory and statistics underpinning phylogenetic analysis.
 \item Apply this theory to computational phylogenetic analyses and learn to use remote computing resources (MERCED cluster)
 \item Assess and discuss primary research papers with a phylogenetic focus
\end{itemize}
\end{frame}



\end{document}