\documentclass{beamer}
\useoutertheme{infolines}
\usepackage[utf8]{inputenc}
\usepackage{default}
%\usepackage{enumitem}
\usepackage{graphicx}
\usepackage[normalem]{ulem}
\newcommand\redout{\bgroup\markoverwith
{\textcolor{red}{\rule[.5ex]{2pt}{2pt}}}\ULon}
\usepackage{xcolor}
\usepackage{textpos}
\usepackage{tikz}   
\usepackage[round]{natbib}
\bibliographystyle{apalike}
\usepackage[T1]{fontenc}
\usepackage{pdfpages}
\usepackage{hyperref}


%\addtobeamertemplate{frametitle}{}{%
%\begin{textblock*}{100mm}(.8\textwidth,-1cm)
%\includegraphics[height=1cm,width=2cm]{opentreelogogrey}
%\end{textblock*}}

\useinnertheme{rectangles}
\usetheme{Copenhagen}
\setbeamertemplate{footline}[frame number]{}
\setbeamertemplate{footline}{}

\title[*]{Homology and inference}
\author[*]{Emily Jane McTavish}
\institute[*]{
Life and Environmental Sciences\\
University of California, Merced\\
\texttt{ejmctavish@ucmerced.edu, twitter:snacktavish}\\
}
\date{}
\setbeamertemplate{itemize items}[triangle]

\begin{document}

\begin{frame}
\titlepage
(With thanks to Mark Holder and Richard Edwards for slides) 
\end{frame}


\setbeamercolor{background canvas}{bg=}
\includepdf[pages=13-16]{/home/ejmctavish/Desktop/old_classes/QSB244/markphylo/2013/lec3-ParalogyAndInference.pdf}




\setbeamercolor{background canvas}{bg=}
\includepdf[pages=26-27]{/home/ejmctavish/Desktop/old_classes/QSB244/markphylo/2013/lec3-ParalogyAndInference.pdf}


\setbeamercolor{background canvas}{bg=}
\includepdf[pages=1-4]{/home/ejmctavish/Desktop/old_classes/QSB244/markphylo/2013/lec4-Inference.pdf}


\begin{frame}
  \includegraphics[width=\textwidth]{phylo_midpoint-rooting}
\end{frame}

\begin{frame}
 \includegraphics[width=\textwidth]{phylo_outgroup-rooting}
\end{frame}



%-----Digression into newick format

\begin{frame}
 \center{\Large a brief digression into newick tree file format}
\end{frame}


\begin{frame}
\begin{centering}
\includegraphics[height=0.4\textheight]{/home/ejmctavish/Desktop/Talks/newicklobster}
\end{centering}\\
 Newick’s Lobster House was the site of an historic 1986 meeting 
 at which a standard was devised for storing descriptions of 
 phylogenetic trees as strings. 
 (Photo from Paul Lewis)
\end{frame}


\setbeamercolor{background canvas}{bg=}
\includepdf[pages=79-91]{/home/ejmctavish/Desktop/Talks/2013_SISG_13_1_slidelayout.pdf}



\begin{frame}
Newick\\
\begin{itemize}
 \item Parenthetical tree format
 \item Rooted vs. unrooted trees are not differentiated
 \item Some programs interpret polytomy at root as `unrooted
 \item Branches and nodes not well differentiated
 \item A name can contain and characters except blanks, colons, semicolons, parentheses, and square brackets
\end{itemize}
\end{frame}


%----------Play Break---------------
\begin{frame}
\frametitle{Exercise}
Create a newick tree file in your text editor with the content:
\texttt{(((C,(D,E)),(F,G),A),B);}\\
Save it as 'example.tre'.\\
\begin{itemize}
 \item Draw the tree by hand
 \item Write down all the splits in ..** format.
 \item Re-root  the tree. What rootings make the following true? Which cannot be true?
 \begin{itemize}
 \item A is more closely related to G than it is to C
 \item (C,D,E) is sister to (A,B,F,G)
 \item (C,D) is sister to (A,B,E,F,G)
 \item (C,D,E) is a paraphyletic group
 \item (C,D,E) is a monophyletic group
 \item (A,B,C) is a monophyletic group
 \end{itemize}
\end{itemize}
\end{frame}




\end{document}
