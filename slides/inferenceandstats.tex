\documentclass{beamer}
\useoutertheme{infolines}
\usepackage[utf8]{inputenc}
\usepackage{default}
%\usepackage{enumitem}
\usepackage{graphicx}
\usepackage[normalem]{ulem}
\newcommand\redout{\bgroup\markoverwith
{\textcolor{red}{\rule[.5ex]{2pt}{2pt}}}\ULon}
\usepackage{xcolor}
\usepackage{textpos}
\usepackage{tikz}   
\usepackage[round]{natbib}
\bibliographystyle{apalike}
\usepackage[T1]{fontenc}
\usepackage{pdfpages}
\usepackage{hyperref}


%\addtobeamertemplate{frametitle}{}{%
%\begin{textblock*}{100mm}(.8\textwidth,-1cm)
%\includegraphics[height=1cm,width=2cm]{opentreelogogrey}
%\end{textblock*}}

\useinnertheme{rectangles}
\usetheme{Copenhagen}
\setbeamertemplate{footline}[frame number]{}
\setbeamertemplate{footline}{}

\title[*]{Homology and inference}
\author[*]{Emily Jane McTavish}
\institute[*]{
Life and Environmental Sciences\\
University of California, Merced\\
\texttt{ejmctavish@ucmerced.edu, twitter:snacktavish}\\
}
\date{}
\setbeamertemplate{itemize items}[triangle]

\begin{document}

\begin{frame}
\titlepage
(With thanks to Mark Holder for slides) 
\end{frame}


\setbeamercolor{background canvas}{bg=}
\includepdf[pages=48-55]{/home/ejmctavish/Desktop/old_classes/QSB244/markphylo/2017/lec1-IntroStats.pdf}

\setbeamercolor{background canvas}{bg=}
\includepdf[pages=57-63]{/home/ejmctavish/Desktop/old_classes/QSB244/markphylo/2017/lec1-IntroStats.pdf}

\setbeamercolor{background canvas}{bg=}
\includepdf[pages=70]{/home/ejmctavish/Desktop/old_classes/QSB244/markphylo/2017/lec1-IntroStats.pdf}



\setbeamercolor{background canvas}{bg=}
\includepdf[pages=72-74]{/home/ejmctavish/Desktop/old_classes/QSB244/markphylo/2017/lec1-IntroStats.pdf}


%\appendix
%\begin{frame}[allowframebreaks]
% \bibliography{GradPhylo}
%\end{frame}

\begin{frame}
Rule: Two taxa that share a character state must be more
closely related to each other than either is to a taxon that
displays a different state.(method suggested by Hennig)\\ 
\textit{Is this a valid rule?}
\end{frame}
%\appendix
%\begin{frame}[allowframebreaks]
% \bibliography{GradPhylo}
%\end{frame}


\setbeamercolor{background canvas}{bg=}
\includepdf[pages=75-80]{/home/ejmctavish/Desktop/old_classes/QSB244/markphylo/2017/lec1-IntroStats.pdf}


\begin{frame}
 \includegraphics[width=\textwidth]{noconflictmat}\\
Draw the rooted tree and unrooted trees consistent with these data.\\
Mark character state changes on the tree.
\end{frame}

\setbeamercolor{background canvas}{bg=}
\includepdf[pages=82]{/home/ejmctavish/Desktop/old_classes/QSB244/markphylo/2017/lec1-IntroStats.pdf}


\begin{frame}
\begin{itemize}
 \item If characters are not polarized (ancestral and descendent states known)
 method can infer unrooted trees.
 \item We can infer tree topology, but be unable to tell paraphyletic from
monophyletic groups.
 \item The outgroup method amounts to inferring an unrooted tree and then
rooting the tree on the branch that leads to an outgroup.
\end{itemize}
\end{frame}


\setbeamercolor{background canvas}{bg=}
\includepdf[pages=4]{/home/ejmctavish/Desktop/old_classes/QSB244/markphylo/2017/lec4-ML-CompatPars.pdf}

\begin{frame}
\textbf{problems with this approach}
\begin{itemize}
 \item We don't know polarization
 \item We observe character conflict in real data sets
\end{itemize}
\end{frame}


\setbeamercolor{background canvas}{bg=}
\includepdf[pages=6]{/home/ejmctavish/Desktop/old_classes/QSB244/markphylo/2017/lec4-ML-CompatPars.pdf}

\begin{frame}
\textbf{Character conflict}
\begin{itemize}
 \item Two characters are compatible if they can both be mapped on the same
tree so that all of the character states displayed could be homologous.
 \item Incompatible characters are evidence of homoplasy in the data
 \item Homoplasy literally means the “same change” has occurred more than once
in the evolutionary history of the group.
\item The presence of homoplasy undermines analyses which rely on counting and minimziing changes.
\end{itemize}
\end{frame}


\begin{frame}
 \includegraphics[width=\textwidth]{conflictmat}\\
Draw the a tree consistent with these data. Mark character state changes on the tree.
\end{frame}


\begin{frame}
 \includegraphics[width=\textwidth]{conflictmat}\\
How many trees can be drawn with the same number of character state changes for these data?\\
What factors would make each of those alternative trees seem more or less likely?
\end{frame}


\begin{frame}
\textbf{Should we expect character conflict?}
\begin{itemize}
 \item Data type?
 \item Evolutionary history?
\end{itemize}
\end{frame}

\begin{frame}
\textbf{Parsimony}
\begin{itemize}
 \item The simplest explanation is the most likely to be true
 \item Applied to phylogenetic inference, it is the inference metric that the tree with the fewest state changes is correct
\end{itemize}


\end{frame}

\setbeamercolor{background canvas}{bg=}
\includepdf[pages=1-5]{/home/ejmctavish/Desktop/old_classes/QSB244/markphylo/parsSummaryModels.pdf}



\begin{frame}
\textbf{How can we deal with character conflict?}
\begin{itemize}
 \item We need to apply an error model
 \item Likelihood provides a measure of surprise under different models
\end{itemize}
\end{frame}


\setbeamercolor{background canvas}{bg=}
\includepdf[pages=15-20]{/home/ejmctavish/Desktop/old_classes/QSB244/Lewis/Likelihood2017.pdf}



\end{document}
