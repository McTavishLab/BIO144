\documentclass[a4paper,10pt]{article}
%\documentclass[a4paper,10pt]{scrartcl}

\usepackage[utf8]{inputenc}

\title{}
\author{}
\date{}

\pdfinfo{%
  /Title    ()
  /Author   ()
  /Creator  ()
  /Producer ()
  /Subject  ()
  /Keywords ()
}

\begin{document}
\maketitle

Lab 3:

Modified from http://carrot.mcb.uconn.edu/~olgazh/bioinf2010/class29.html


Long Branch Attraction (LBA) is a serious problem in phylogenetic reconstruction. 
LBA denotes the fact that long branches tend to be grouped together with significant support, 
even though the organisms representing the long branches did not share more recent common ancestry. 
The support usually is measured through bootstrap support values for the different trees. 
The evolution of four sequences (named A,B,C,D) was simulated according to the following tree:

\includegraphics{LBA_tree}\\

Files containing these sequences in multiple sequence fasta format were generated and named according to the length (x) 
chosen for the two long branches (all scaled in substitutions per site). 
For the simulation we assumed that the among-site-rate-variation could be described with a gamma distribution that has a shape parameter of 1 (equivalent to an exponential distribution).

These files are HERE. 

Save the files sequences in the directory where the PHYLIP executables are stored (within the "phylip-3.69/exe" folder).



\end{document}
